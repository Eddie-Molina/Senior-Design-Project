<<<<<<< HEAD
The user will have full control of the brewing processes using the given software. This software will have a UI that can be used on both desktops and mobile devices to offer the user an easy experience to brewing. During use of the product, the user will be notified of the current status of the product and be able to continue or stop at any time in case of emergencies. 

\subsection{Custom Recipes}
\subsubsection{Description}
There will be a function in the UI to allow a user to upload a .txt file containing a recipe. 
\subsubsection{Source}
Customer
\subsubsection{Constraints}
The recipe must be in a specified format which will be determined at a later date. 
\subsubsection{Standards}
N/A
\subsubsection{Priority}
Low

\subsection{Info Tab}
\subsubsection{Description}
The UI should be simple to use and contain the necessary information on the first tab. Information includes time left on the entire process and temperature. More information will de decided on a later date. 
\subsubsection{Source}
Customer
\subsubsection{Constraints}
N/A
=======
%Include a header paragraph specific to your product here. Customer requirements are those required features and functions specified for and by the intended audience for this product. This section establishes, clearly and concisely, the "look and feel" of the product, what each potential end-user should expect the product do and/or not do. Each requirement specified in this section is associated with a specific customer need that will be satisfied. In general Customer Requirements are the directly observable features and functions of the product that will be encountered by its users. Requirements specified in this section are created with, and must not be changed without, specific agreement of the intended customer/user/sponsor.

The D.O.O.M. bot will automactically add hops to a boil at the apprioprate times per a preprogrammed recipe in the beersmith 2 XML format. It should be able to handle this without human supervision and intervention once the boil process has begun. 

\subsection{Add hops at specific times according to a recipe}
\subsubsection{Description}
The bot will add hops as the customer requires. It will be able to add any kind of hops that the customer pre-loads the device with. It will also remove and rinse the hops using a secondary storage tank for water and a pump.
\subsubsection{Source}
The Brew Crew
\subsubsection{Constraints}
The D.O.O.M. bot can not detect what kind of hops are in each bag, or if the bags are even loaded with hops. 
\subsubsection{Standards}

\subsubsection{Priority}
This is critical to the success of the project. Without this requirement nothing would happen.

\subsection{Read the beersmith 2 XML format and execute based on that}
\subsubsection{Description}
The device should be able to parse beersmith2 xml files and collect the times in which hops are to be added. It should be able to do this with any valid file using the beersmith2 xml format.
\subsubsection{Source}
The Brew Crew
\subsubsection{Constraints}
Everything about this is predicated on the idea that the recipe is correct in the XML file, if no time exists there is nothing the machine can do. 
>>>>>>> master
\subsubsection{Standards}
N/A
\subsubsection{Priority}
<<<<<<< HEAD
Critical

\subsection{iOS and Android Helper Application}
\subsubsection{Description}
The software will also be able to run on both iOS and Android alongside its desktop counterpart. It will be able to connect to the system via a local wifi network. 
\subsubsection{Source}
Customer
\subsubsection{Constraints}
N/A
\subsubsection{Standards}
802.11g/b/a
\subsubsection{Priority}
Moderate
=======
This also has critical priority and is fundemental to the success of the project. Without this the device can not perform it's intended function.
>>>>>>> master
